\documentclass{beamer}

\usetheme{Warsaw}

\title[Virtual Machines]
{
    What Are Virtual Machines and How Can We Use Them
}

\author[Kolovic]
{
    M.~Kolovic
}

\institute[UWO]
{
    Robarts Research Institute \\
    University of Western Ontario
}

\date[Summer 2016]
{
    Data Club \\
    \today
}

\subject{virtual machines, practical computing}


\begin{document}

\section[Outline]{Outline}

\begin{frame}
    \titlepage
\end{frame}

\begin{frame}
    \frametitle{Outline}
    \tableofcontents
\end{frame}

\begin{frame}
    \frametitle{Cut to the chase!}
    \begin{itemize}
        \item turn one computer into many
        \item last software you will ever install...
        \item protect your computer, expand its capabilities
    \end{itemize}
\end{frame}

\AtBeginSection[]
    {
        \begin{frame}
            \tableofcontents[currentsection]
        \end{frame}
    }
\section[Motivation]{Why should we care?}

\begin{frame}
    \frametitle{My computer is fine}
    \begin{itemize}
        \item you're running Mac or Windows, called your Host 
        \item games, music, video, pictures, web
        \item<2-> what if you want to run something more complex?
            \begin{itemize}
                \item<3-> OS incompatabilities
                \item<3-> dependency nightmares
                \item<3-> break system, reinstall
            \end{itemize}
    \end{itemize}
\end{frame}

\begin{frame}
    \frametitle{You wouldn't do lab work at home}
    \begin{itemize}
        \item experimenting on your Host will cause you headaches
        \item save time with fully isolated environments
        \item turn a bit of our home into a lab safely, cleanly, securely
    \end{itemize}
\end{frame}

\begin{frame}
    \frametitle{What can we do in a VM?}
    \begin{itemize}
        \item save money, resources, fully utilize hardware
        \item test multiple OS simultaneously, no restart
        \item protect your Host from unverified programs
        \item backup with the click of a button
    \end{itemize}
\end{frame}

\begin{frame}
    \frametitle{Do I have to use a VM?}
    \begin{itemize}
        \item get new hardware
            \begin{itemize}
                \item impractical
                \item expensive
                \item good luck maintaining
            \end{itemize}
        \item dual boot
            \begin{itemize}
                \item reboot to switch
                \item usual maintenance tasks
                \item partition hard drive
            \end{itemize}
    \end{itemize}
\end{frame}

\begin{frame}
    \frametitle{They're already around us}
    \begin{itemize}
        \item Hegele lab uses virtualized servers
            \begin{itemize}
                \item process LipidSeq data
                \item NextSeq analysis 
            \end{itemize}
        \item some of our desktops too
    \end{itemize}
\end{frame}

\section[Basics]{What are Virtual Machines?}

\begin{frame}
    \frametitle{What are Machines?}
    \begin{itemize}
        \item computing devices composed of four core pieces
            \begin{description}[CPU]
                \item[CPU] the central processing unit performs all computational tasks
                \item[RAM] random access memory provides data storage that is readily accessible
                \item[Storage] HDD or SSD for permanent data storage
                \item[Network] to communicate with other computing devices
            \end{description}
        \item OS saved on Storage, loaded into RAM on boot, interacts with CPU directly
        \item "bare metal"
    \end{itemize}
\end{frame}

\begin{frame}
    \frametitle{What is Virtual?}
    \begin{itemize}
        \item "not physically existing as such but made by software to appear so"
        \item OS will not directly interact with hardware
        \item hardware virtualization
            \begin{itemize}
                \item Guest OS acts like interacting with real hardware
                \item hypervisor "fakes" it, safe instructions executed directly on CPU, privileged requests translated and taken care of by hypervisor 
                \item this is "full" hardware virtualization
            \end{itemize}
        \item hardware-assisted virtualization
            \begin{itemize}
                \item Intel VT-x or AMD-V
                \item Guest can execute privileged instructions, CPU has built in functions to help hypervisor handle these
            \end{itemize}
    \end{itemize}
\end{frame}

\section[Hands On]{How can I get started?}

\begin{frame}
    \frametitle{Is my computer capable?}
    \begin{itemize}
        \item check if cpu supports virtualization
        \item how many cores do you have?
        \item do you have sufficient hard disk space?
        \item how much ram can you allocate?
    \end{itemize}
\end{frame}

\begin{frame}
    \frametitle{How do I do it?}
    \begin{itemize}
        \item many virtualization softwares available
            \begin{itemize}
                \item Parallels
                \item VMWare
                \item VirtualBox
            \end{itemize}
        \item VirtualBox is very popular, free, and available on most OS
        \item many operating systems available
            \begin{itemize}
                \item Ubuntu, Kubuntu, Lubuntu
                \item Windows or Mac if you have install disks
            \end{itemize}
    \end{itemize}
\end{frame}

\section[Summary]{Summary}

\begin{frame}
    \frametitle{Could you repeat that?}
    \begin{itemize}
        \item simplify your life
        \item protect your Host
        \item liberate your resources
        \item doorway tool and skill
    \end{itemize}
\end{frame}

\begin{frame}
    \begin{center}
        Thank You
    \end{center}
\end{frame}

\end{document}
